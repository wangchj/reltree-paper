\section{Introduction}\label{sec-intro}
In the participatory sensing, users with mobile devices (e.g. cell phones
and laptop computers) participate in the collection of environmental
information around them and submit the collected data to a central server
for processing and analysis (Estrin). An advantage of this computing model
is that dynamic environmental data are quickly captured by mobile users
and disseminated to other users or central servers for analysis.

Other ongoing research has been conducted in the areas where participatory
model is useful.

When a mobile user participates in data collection, the user's identity is usually associated with the captured data submitted to the server for
storage.  As address by Ling et al. in "Privacy Assurance in Mobile Sensing Networks: Go Beyond Trusted Servers" a major concern with this model is
identity privacy of the users. For example, if a user captured a controversial
piece of information, the user might be reluctant to submit the data
to the server and therefore associate his or her information with such data.

In this writing, we propose a way to protect the identity of participants
of such computing model by using a novel(?) anonymous networking protocol.
The goal of the protocol is to remove the source information from the data
packet. We call this One-Way Protocol since the source information is
complectly removed, and the server is unable to reply any kind of
acknowledgement back to the client. The protocol will be implemented on
both the server and the client to achieve the anonymous effect.

In Section 2, we propose a fundamental theoretical novel One-Way protocol
that probably has no practical usefulness except to demonstrate the idea.
(In Section X we present other solutions to this problem). In section 3
we refine the theoretical protocol so that it can be architectured on
existing networking infrastructures and technologies. Validation and data
integrity is covered in section 4. 