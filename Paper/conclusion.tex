\section{Conclusions}\label{sec-conc}

In this paper we proposed One-Way, an anonymous network protocol designed
for the mobile, participatory sensing environment and provides identity
privacy to mobile users (data senders). Constraints of bandwidth and
computational capability were the biggest motivations for our design.
Our protocol utilizes peer-to-peer network to construct an anonymous
route on which control messages (such as data acknowledgement) can be sent
to facilitate data transmission. On the peer route, each node keeps track
of its predecessor and successor, but does not know the length of the,
thus keeping the identity of the sender anonymous. In order to conserve
bandwidth in mobile environment, the actual data payload is not sent
through peers, but instead send through a direct path using the One-Way
protocol. In One-Way protocol, payload messages are sent to the server
without the sender's identity (or address). When a server receives the
payload, it sends an acknowledgement back to the sender through the
anonymous peer-to-peer route.

We presented two implementations for our protocol: using trusted gateway
and on top of UDP transport layer protocol. Trusted gateway is useful
when the ISP has stringent policy of correct sender address. In the case
of implementing on top of UDP, we use the broadcast address to replace
the sender's address to signify that the packet should be treated as
anonymous.

Our performance evaluations show that our protocol performs well comparing
with other privacy protocols in terms of resource utilizations.
Our protocol incurs significantly less transmission overhead (in
terms of transmission bytes) due to the fact that we only utilize peer
network for small portion of the transmission. Transmission latency is
also improved over other protocols. For future work, we would like to
perform further analysis of our design, such as scalability and practical
implementation issues and adaptability by ISPs. We would also like to
further study security issues and study more attack models for our design.