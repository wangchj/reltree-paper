
In this section we consider several attack models and discuss how
our design offers protection against these attacks.

\subsection{Server Attack}
The first attack model is server attack. In this attack, the
server cannot be trusted or can be a source of identity of
disclosure. Most services log the identity of users with the data
submitted through their services. If the server is malicious, or
mandated by someone powerful for disclosure, privacy of users is
compromised.

First, we analyze the payload transmission part of our design
against server attack. We assume that the server can see the
packets sent to it, but does not see the packets in the network
before they reach it. In other words, the server does not control
the Internet and cannot perform traffic analysis on the Internet.
Under this assumption, in our One-Way protocol the sender has the
anonymity level of beyond suspicion since a server cannot infer
any identity information from the data packets it receives. The
only linking information is the identity of the network device
(e.g., link layer switch) at the last hop of the route. We assume
the service does not control the Internet, therefore it cannot
link this information back to the sender.

Next, we analyze our validation protocol against server attack.
Keep in mind that our validation mechanism utilizes peer network
to transmit acknowledgements and control messages. Since the
server has to execute our anonymous validation protocol, it is
considered one of the network peers and therefore it can find out
the users in the peer network, but the server does not know the
original hop count $h$ of the control packets. In addition, any
peer can potentially be connected with any other peers.
Consequently, the sender is also beyond suspicion in this case.

\subsection{Eavesdropper and Traffic Analysis}

In eavesdropper attack, we assume that the data transmission route
is divided into segments each controlled by a network
administrator. An attacker is able to listen to the traffic of a
segment, but cannot eavesdrop on all the network segments of the
route. In other words, the attacker does not have the full view of
the entire route, but only a segment.

There are two scenarios. The first scenario is that the sender of
the data does not reside on the network on which the attacker is
eavesdropping and the the level of anonymity for the sender is
beyond suspicion. Although the attacker is able to listen to the
traffic to her network, she is unable to see who the sender is in
other network segments.

The second scenario is that the sender and the attacker are on the
same network. In this case the attacker is able to analyze the
traffic of any node on the network. In this scenario, our design
is unable to protect the identity of the sender and the level of
anonymity is exposed. The way that the attack could get the
sender's identity is to compare the incoming and outgoing traffic
for every node in the network. For the nodes that have outgoing
traffic for a specific content without corresponding incoming
traffic is suspected of data originator. Encryption of data on top
of our design can mitigate this attack.
