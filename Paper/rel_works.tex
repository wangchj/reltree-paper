
One of the earliest work on anonymous communication is on
untraceable electronic mail by
Chaum~\cite{DBLP:journals/cacm/Chaum81}. The work proposes a way
of sending an electronic mail to a receiver in which the sender
remains anonymous using one or a series of proxy computers called
\emph{mix}. Chaum's solution relies heavily on public key
cryptography in a way that before forwarding a mail to a mix, the
message and the receiver's address is encrypted using the mix's
public key. Upon receiving an electronic message, the mix extracts
the information and forwards the mail to the intended receiver. In
the case of multiple mixes, the sender performs one public key
cryptography for each mixes before sending a mail to the first
mix. The mix then forward it to the next mix and so on until the
mail reaches the receiver. One of the drawbacks of the solution
proposed in~\cite{DBLP:journals/cacm/Chaum81} is scalability since
the senders depend on fixed proxies and if there are no proxies,
the solution will not work. Also multiple public key cryptography
is expensive for mobile devices. Multiple public key encryptions
is acceptable for small files, but for large files that need to be
broken down and sent with multiple segments, multiple encryptions
may not be acceptable.

Another well known research of sender anonymity is the onion
routing\cite{DBLP:conf/uss/DingledineMS04}. The onion routing
follows the similar approach proposed by Chaum of using multiple
proxies and multiple encryptions. Before sending a file, the
sender selects a number of proxies called onion routers, which
mitigates malicious node collaboration attacks. After selecting
the routers, the sender encrypts the data with the key of the
first router, then encrypts the result of that with the key of the
second router, and does the same for all the rest routers. Onion
routing is robust in terms of security and can be used in many
applications, but it has the same drawback as in Chaum's approach
that it is computationally expensive because of the multiple
encryptions. Our solution does not impose multiple encryptions to
the application, but leaves that choice to the developer of the
application.

Freenet, a distributed anonymous storage system was proposed
in~\cite{DBLP:conf/diau/ClarkeSWH00}. The goal of the system is to
utilize the peers in the network for storage and at the same time
achieving anonymity of the authorship of the files. Freenet
employs file keys obtained by public key cryptography and hashing
to achieve efficient file storage and searching. In order to
achieve author anonymity, a node on the Freenet sends a file to a
peer, and the peer forwards the file to another peer and so on.
Although, the system pertains to file storage, the technique of
utilizing peers to hide the identity of the originator of a file
is useful in our scenario and we take a similar approach with data
verification of our design.

Crowds~\cite{DBLP:journals/cacm/ReiterR99} utilizes peer-to-peer
networks to blend the sender into a large number of users (a
crowd) so that the receiver does not know the true identity of the
sender. To send a file, the sender joins a close-by crowd and
sends the file to one of her peers in the crowd called jondo (a
short name for John Doe). When a peer receives a send request, she
decides whether to forward the data to another peer based on the
variable $p_f$ that determines the probability of forwarding the
packet.

The closest work to ours is the research on privacy assurance for
mobile users in sensing networks by Hu and
Shahabi~\cite{DBLP:conf/percom/HuS10}. The assumption of the work
is that data collection service providers often log the identities
of users with the file they submit. The disclosure of this
sensitive information leads to the compromising of user privacy.
The solution is to leverage social network to forward a file to
multiple peers before sending it to the server. Large consumption
of network bandwidth is a big disadvantage in this approach since
a file is sent to multiple users, and this concern is exacerbated
in wireless mobile environments where bandwidth is limited. In our
approach, we do not forward the payload of the data to multiple
peers, only the control messages of the data transmission and
saves tremendous amount of bandwidth.
