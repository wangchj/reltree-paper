\section{Introduction}\label{sec-intro}
In the participatory sensing, users with mobile devices
(e.g. cell phones and GPS devices) participate in the collection of environmental
information around them and submit the collected data to a central server
for processing, analysis, and storage\cite{DBLP:conf/mobisys/MunRSYBEHHWB09}.
An example is the OpenStreetMap\cite{DBLP:journals/pervasive/HaklayW08}
project which collects user contributed GIS
data (such as GPS traces and POIs) and constructs an online map, from mobile
data, open for anyone to use. 
Other examples of participatory sensing
include cooperative collection of gas prices \cite{DBLP:conf/dcoss/DongKCB08},
monitoring of traffic conditions \cite{DBLP:conf/dcoss/ConceicaoFB08}, and
photo sharing through mobile applications such as Facebook and Instagram.

This model of "crowd" data collection has many advantages due to that the
dynamic environmental data and information are quickly captured by mobile
users and disseminated to other users or to central servers for analysis.
Government censoring of information captured and distributed is also less
effective in this model since there could be many mobile devices that are
capturing the same event.

Despite the advantages, privacy of the users who contribute is a major
issue in participatory sensing and crowd-sourcing. Users may not want to
leak their locations with the contributed data lest their movement being
tracked. In the case of controversial photos, a user may not want to
reveal his or her identity to the server (such as
Instagram\footnote{http://instagr.am/}) that collect
the data since the user's identity may be saved on the server and later
uncovered by government entities for prosecution. It is a fact that
services on the web collect user identifications along with the data
submitted by their users\cite{wiki_privacy}. Even with websites, such as
Wikipedia\footnote{http://www.wikipedia.com} and
4chan\footnote{http://www.4chan.org/faq\#what4chan}, that advise to be
anonymous log IP address of contributors. With
an IP address (and if IP uniquely identify a user), a piece of
information stored on a server (be it a photo, or a file) can be traced
back to a user thus compromise the user's privacy.

How to guarantee that a server will not gather users identification when
a service advertise it? In this paper, we focus on the problem of keeping
user's identity private
in participatory sensing and crowd-sourcing environment. We want to keep
a request to a server anonymous to all entities in a network including
the server itself (since we know they log user activities).
We study a way to
minimize the probability of identifying and linking a piece of data
in a computer network to a user.
Then we propose a way to protect the identity of participants
of such computing model by using an anonymous networking protocol.
The goal of the protocol is to remove the source information from the data
packet. We call this One-Way Protocol since the source information is
completely removed, and the server is unable to reply any kind of
acknowledgement back to the client. The protocol will be implemented on
both the server and the client to achieve the anonymous effect.

%In Section 2, we propose a fundamental theoretical novel One-Way protocol
%that probably has no practical usefulness except to demonstrate the idea.
%(In Section X we present other solutions to this problem). In section 3
%we refine the theoretical protocol so that it can be architectures on
%existing networking infrastructures and technologies. Validation and data
%integrity is covered in section 4. 