\section{Introduction}\label{sec-intro}
In the participatory sensing, users with mobile devices
(e.g. cell phones and GPS devices) participate in the collection of environmental
information around them and submit the collected data to a central server
for processing, analysis, and storage\cite{DBLP:conf/mobisys/MunRSYBEHHWB09}.
An example is the OpenStreetMap\cite{DBLP:journals/pervasive/HaklayW08}
project which collects user contributed GIS
data (such as GPS traces and POIs) and constructs an online map, from mobile
data, open for anyone to use.
Other examples of participatory sensing
include cooperative collection of gas prices\cite{DBLP:conf/dcoss/DongKCB08},
monitoring of traffic conditions\cite{DBLP:conf/dcoss/ConceicaoFB08}, and
photo sharing through mobile applications such as Facebook and Instagram.

This model of "crowd" data collection has many advantages due to that the
dynamic environmental data and information are quickly captured by mobile
users and disseminated to other users or to central servers for analysis.
Government censoring of information is also less
effective in this model since there could be many mobile devices that are
capturing the same event.

Despite the advantages, privacy of the users who contribute is a major
issue in participatory sensing and crowd-sourcing. Users may not want to
leak their locations with the contributed data lest their movement being
tracked. In the case of controversial photos, a user may not want to
reveal his or her identity to the collection server (such as
WikiLeaks\footnote{http://wikileaks.org/})
since the user's identity may be saved on the server and later
uncovered by government entities for prosecution. It is a fact that
services on the web collect user identifications along with the data
submitted by their users\cite{wiki_privacy}. Even with websites, such as
Wikipedia\footnote{http://www.wikipedia.com} and
4chan\footnote{http://www.4chan.org/faq\#what4chan}, that advertise to be
anonymous, log IP address of contributors.
Although IP address does not uniquely identify an individual, cellular
data service providers are required by law to keep record of assignment of
IP addresses to mobile users for a period of time\footnote{http://www.law.cornell.edu/uscode/18/usc\_sec\_18\_00002703----000-.html}.
With an IP address, a piece of
information stored on a server (be it a photo, or a file) can be traced
back to a user using the technique discussed in \cite{DBLP:journals/ijufks/Sweene02}
thus compromise the user's privacy.

How to guarantee that a server will not gather users identification when
a service advertise it? Many previous works
\cite{DBLP:conf/uss/DingledineMS04} \cite{DBLP:journals/cacm/ReiterR99}
\cite{DBLP:journals/jcs/LevineS02} \cite{DBLP:conf/percom/HuS10}
(discussed in the next section)
have addressed the issues of user anonymity on the Internet. Unfortunately,
most previously proposed solutions are designed for the older computing model
of desktop computing where network bandwidth and computing power are not major concerns.
In mobile environment, both bandwidth and computational capability are limited
partially due to the size of devices and battery power limitations.

In this paper, we propose an anonymous network protocol with constraints
of mobile, participatory sensing environment in mind. The goal is to design a
solution that is simple so that it does not consume to much resource for
the mobile devices (resources being bandwidth and computational power),
at the same time provides adequate protection for privacy.
We want to keep the identity of data originators (senders) private to
various parties in a computer network, especially to the data collection
server(receiver) since without identity information associated with submitted
data on the server, linking to originator is difficult. Although we focus
on keeping sender anonymous to the server, we believe our solution has enough
protection from other parties in a network.
Our protocol is to remove the source information from the data
packet. We call this One-Way Protocol since the source information is
completely removed, and the server is unable to reply any kind of
acknowledgement back to the client. The protocol will be implemented on
both the server and the client to achieve the anonymous effect.

In section~\ref{sec-related-works}, we discuss previous works for anonymity
on computer networks.
We each work, we identify strengths and weaknesses relative to our design
goals.
In Section~\ref{sec-protocol}, we present our anonymous One-Way protocol.
We discuss how data payload is transmitted and how data is verified although
the server does not know the identity of the sender.
Section~\ref{sec-imp} presents implementation issues.
Section~\ref{sec-attack} discusses a few attack models.
Section~\ref{sec-evaluation} presents our performance evaluations.
Section~\ref{sec-conc} concludes our paper.

%that probably has no practical usefulness except to demonstrate the idea.
%(In Section X we present other solutions to this problem). In section 3
%we refine the theoretical protocol so that it can be architectures on
%existing networking infrastructures and technologies. Validation and data
%integrity is covered in section 4. 