
In participatory
sensing~\cite{conf/sensys/ReddyBEHS07,conf/sensys/DuttaAKMMWW09},
users with mobile devices (e.g., cell phones with GPS chips)
participate in the collection of environmental information around
them and submit the collected data (image, audio, video, etc.) to
a central server for process, analysis, and
storage~\cite{DBLP:conf/mobisys/MunRSYBEHHWB09}. An example is the
OpenStreetMap~\cite{DBLP:journals/pervasive/HaklayW08} project,
which collects user contributed GIS data (such as GPS traces and
POIs) and constructs an online map, from mobile data, open for the
public to use. Other examples of participatory sensing include
cooperative collection of gas
prices~\cite{DBLP:conf/dcoss/DongKCB08}, monitoring of traffic
conditions~\cite{DBLP:conf/dcoss/ConceicaoFB08}, and photo sharing
through mobile applications such as Facebook and Instagram.

This model of ``crowd" data collection has many advantages due to
that the dynamic environmental data and information are quickly
captured by mobile users and disseminated to other users or to
central servers for analysis. Government censoring of information
is also less effective in this model since there could be many
mobile devices that are capturing the same event.

Despite the advantages, privacy of the users who contribute is a
major issue in participatory sensing and crowd-sourcing. Users may
not want to leak their locations with the contributed data lest
their movement should be
tracked~\cite{conf/icde/YiuJHL08,journals/pvldb/WangL09}. In the
case of controversial photos, a user may not want to reveal his or
her identity to the collection server (such as
WikiLeaks\footnote{http://wikileaks.org/}) since the user's
identity may be saved on the server and later uncovered by
government entities for prosecution. It is a fact that services on
the web collect user identifications along with the data submitted
by their users~\cite{wiki_privacy}. Even with websites, such as
Wikipedia\footnote{http://www.wikipedia.com} and
4chan\footnote{http://www.4chan.org/faq\#what4chan} that advertise
to be anonymous log IP addresses of contributors. Although an IP
address does not uniquely identify an individual, cellular data
service providers are required by law to keep record of assignment
of IP addresses to mobile users for a period of
time\footnote{http://uscode.house.gov/download/pls/18C121.txt}.
With an IP address, a piece of information stored on a server (be
it a photo, or a file) can be traced back to a user using the
technique discussed in~\cite{DBLP:journals/ijufks/Sweene02} thus
compromises the user's privacy.

How to guarantee that a service provider will not gather users'
identification? Many previous
works~\cite{DBLP:conf/uss/DingledineMS04,DBLP:journals/cacm/ReiterR99,
DBLP:journals/jcs/LevineS02} have addressed the issues of user
anonymity on the Internet. Unfortunately, most previously proposed
solutions are designed for the older computing model of desktop
computing where network bandwidth and computing power are not
major concerns. In mobile environments, both bandwidth and
computational capability are limited partially due to the size of
devices and battery power constraints.

In this paper, we propose an anonymous data sharing method with
constraints of mobile, participatory sensing environments in mind.
The goal is to design a solution that is simple so that it does
not consume to much resource for mobile devices (resources being
bandwidth and computational power), and at the same time provide
adequate protection for privacy. We want to keep the identity of
data owners (senders) private to various parties in a computer
network, especially to the data collection server (receiver) since
without identity information associated with submitted data on the
server, linking to data originator is difficult. Although we
mainly focus on keeping sender anonymous to the server, we believe
our solution has enough protection from other parties in a
network. Our mechanism is to remove the source information from
data packets. We call this One-Way Protocol since the source
information is completely removed, and the server is unable to
reply any kind of acknowledgement back to the client. The protocol
will be implemented on both the server and the client to achieve
the anonymous effect.

The rest of this paper is organized as follows. In
section~\ref{sec-related-works}, we survey previous works for
anonymity on data communication. For each work, we identify
strengths and weaknesses relative to our design goals. We present
our anonymous One-Way protocol in Section~\ref{sec-protocol}. We
discuss how data payload is transmitted and how uploaded data is
verified although the server does not know the identity of the
sender. Section~\ref{sec-imp} presents the implementation issues
and Section~\ref{sec-attack} discusses a few attack models. The
experimental validation of our design in
Section~\ref{sec-evaluation}. Section~\ref{sec-conc} concludes
this paper with a discussion of future work.
